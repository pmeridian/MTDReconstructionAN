\section{Summary}
The performance of the full simulation and reconstruction of the MTD detector as implemented for the Technical Design Report have been shown. 

Hit occupancies can be kept below 8\% and 4\% at PU200 with the granularity implemented in the current design for the BTL and ETL respectively. Time resolution of around 35ps can be achieved for the reconstruction of both charged and neutral particles for the time back-propagated to the vertex for events with 200 pile-up interaction vertices, corresponding to the performance of the detector at around 1000~$fb^{-1}$.

Geometric efficiency of ~90\% in the BTL and nearly 100\% in the ETL can be achieved; charged hadron reconstruction efficiency is 5-10\% lower due to nuclear interactions; part of the efficiency loss can be recovered with improvements to the tracking and the MTD association algorithm. 

Using a beam spot constraint for the MTD-track association allows to reduce the fake MTD hit association below 10\% at PU200, with a reduction of a factor 2-5 depending from track $\eta$ and $p_{T}$ with respec to a simple spatial matching algorithm. This number can be further reduced to a level which is close to the 0PU condition when using a vertex constraint. Both association efficiency and fakes can be further improved with additional developments of the track reconstruction at PU200 focused on low $p_{T}$ hadrons. 

4D vertex reconstruction algorithm has been shown to be able to work in PU200 conditions including the effect at low $p_{T}$ of hadrons heavier than pions (kaons, protons) using an iterative approach where low $p_{T}$ particle-identification is performed as an intermediate step. Vertex time can be reconstructed with precision of better than 10~ps for vertices with at least 30-40 tracks associated. 

With a resolution of 35ps per single track and 4D vertex reconstruction, the fraction of tracks due to pile-up wrongly associated to the primary vertex can be reduced of a factor greater than 2. This performance is in fair agreement with the prediction obtained from a fast simulation of the time assignment at the vertex as done for the MTD Technical Proposal for comparable time resolution and efficiency. This shows that the effects due to low $p_{T}$ kaons and protons, path length back-propagation for low $p_{T}$ tracks and MTD fake hits associations can be maintained with a full simulation and reconstruction of the MTD detector to a level which does not spoil the conclusions obtained in the Technical Proposal. 